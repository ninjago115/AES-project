%% LyX 2.3.7 created this file.  For more info, see http://www.lyx.org/.
%% Do not edit unless you really know what you are doing.
\documentclass[english]{article}
\usepackage[T1]{fontenc}
\usepackage{amsmath}
\usepackage{amssymb}
\usepackage{babel}
\usepackage{lmodern}
\usepackage{geometry}

\geometry{left=25mm, right=15mm}
\geometry{top=15mm,bottom=20mm}

\setlength{\parindent}{15pt}
\begin{document}
\tableofcontents
\newpage

\section{Mathematical Basis}

\subsection{Definitions}
Matrix Operations, Groups, Rings and Fields, Vector
Spaces, Polynomials over a Field, Operations on Polynomials.

\subsubsection{Matrix Operations:}
Scalar multiplication:

\begin{itemize}
\item $a$$\odot$$\begin{bmatrix}b, & c\\
d, & e\\
f, & g
\end{bmatrix}$= $\begin{bmatrix}a\cdot b, & a\text{\ensuremath{\cdot c}}\\
a\cdot d, & a\cdot e\\
a\cdot f, & a\cdot g
\end{bmatrix}$
\end{itemize}

Matrix addition: Entrywise sum in this case $\rightarrow$ both matrixes
have same amount of rows and columns

\begin{itemize}
\item $\begin{bmatrix}a_{1}, & a_{2}\\
a_{3}, & a_{4}\\
a_{5}, & a_{6}
\end{bmatrix}$$\oplus\begin{bmatrix}b_{1}, & b_{2}\\
b_{3}, & b_{4}\\
b_{5}, & b_{6}
\end{bmatrix}$=$\begin{bmatrix}a_{1}+b_{1}, & a_{2}+b_{2}\\
a_{3}+b_{3}, & a_{4}+b_{4}\\
a_{5}+b_{4}, & a_{6}+b_{6}
\end{bmatrix}$
\end{itemize}
Scalar addition: $\rightarrow$ multiply scalar with basis of the matrix and then perform matrix addition

\subsubsection{Abelian Group < G, + >:}


Sets of elements with additional structure; ways of combining elements
to produce another element of the set.

Example: < $\mathbb{Z}n,$+ > this group consist of all the integers
from 0 to n - 1 (and the operation is addition modulo n if it is a
finite group) 

\begin{itemize}
\item  closed: \ensuremath{\forall} a, b \ensuremath{\in} G : a + b \ensuremath{\in}
G 

\item associative: \ensuremath{\forall} a, b, c \ensuremath{\in} G : (a
+ b) + c = a + (b + c)
\item commutative: \ensuremath{\forall} a, b \ensuremath{\in} G : a + b
= b + a 
\item neutral element: \ensuremath{\exists} 0 \ensuremath{\in} G, \ensuremath{\forall}
a \ensuremath{\in} G : a + 0 = a 
\item inverse elements: \ensuremath{\forall} a \ensuremath{\in} G, \ensuremath{\exists}
b \ensuremath{\in} G : a + b = 0

\end{itemize}

\subsubsection{Ring < R, +, $\cdot$ >:}


Closed set R with two operations '+' and '$\cdot$'. The set is an
abelian group under < R, + > and the operation '$\cdot$' is an associative
over R and there is a neutral element for '$\cdot$' in R (Usually
1). 
The two operations are related by the law of distributivity:

\begin{itemize}
\item \ensuremath{\forall} a, b, c \ensuremath{\in} R : (a + b) \textperiodcentered{}
c = (a \textperiodcentered{} c)+(b \textperiodcentered{} c)

Depending on if '$\cdot$' is commutative, a Ring can be a \emph{commutative
}or a \emph{noncommutative} Ring.
\end{itemize}

\subsubsection{Field< F, +, $\cdot$ > :}


A set is a Field if < F, +, $\cdot$ > is a commutative Ring and there
exists an inverse element for all F in F with respect to the operation
'$\cdot$', except for the element 0 (Neutral element of < F, + >).

If < F, + > and < F\textbackslash\{0\}, $\cdot$ > are
abelian groups and the law of distributivity applies, the neutral
element of < F\textbackslash\{0\}, \textperiodcentered{} > is called
the \emph{unit element} of the field.

\subsubsection{Vector spaces  < F, V, +, $\oplus$, $\cdot$, $\odot$ >:}


Scalars: Field < F, +, $\cdot$ > with unit element 1.

Vectors: Abelian Group < V, + >

A vector space (also called linear space) < F, V, +, $\oplus$, $\cdot$,
$\odot$ > is a set whose elements, often called vectors, may be added
together and multiplied by numbers called scalars. 

A vector space has to follow these conditions: 
\begin{itemize}
\item Distributivity: \ensuremath{\forall} a \ensuremath{\in} F, \ensuremath{\forall}
v, w \ensuremath{\in} V : a (v+w) = (a v) + (a w) or \ensuremath{\forall}
a, b \ensuremath{\in} F, \ensuremath{\forall} v \ensuremath{\in} V
: (a + b) v = (a v) + (a v)
\item Associativity: \ensuremath{\forall} a, b \ensuremath{\in} F, \ensuremath{\forall}
v \ensuremath{\in} V : (a \textperiodcentered{} b) v = a (b v)
\item neutral element: \ensuremath{\forall} v \ensuremath{\in} V : 1 v =
v
\end{itemize}
Definition the vector operations '$\oplus$' and '$\odot$':
\begin{itemize}
\item $\begin{pmatrix}a_{1}\\
a_{2}\\
\vdots\\
a_{n}
\end{pmatrix}$$\oplus$$\begin{pmatrix}b_{1}\\
b_{2}\\
\vdots\\
b_{n}
\end{pmatrix}$= $\begin{pmatrix}a_{1} & + & b_{1}\\
a_{2} & + & b_{2}\\
\vdots & \vdots & \vdots\\
a_{n} & + & b_{n}
\end{pmatrix}$
\item $a$$\odot$$\begin{pmatrix}b_{1}\\
b_{2}\\
\vdots\\
b_{n}
\end{pmatrix}$=$\begin{pmatrix}a & \cdot & b_{1}\\
a & \cdot & b_{2}\\
\vdots & \vdots & \vdots\\
a & \cdot & b_{3}
\end{pmatrix}$
\end{itemize}
A vector $v$ is a \emph{linear combination} of the vectors $w^{(1)}$,
$w^{(2)}$,$\ldots$ , $w^{(s)}$if there exist scalars $a^{(i)}$such
that :

\begin{itemize}
\item $v$ = $a^{(1)}$$\odot$$w^{(1)}$$\oplus$$a^{(2)}$$\odot$$w^{(2)}$$\oplus$$\cdots$$\oplus$$a^{(i)}$$\odot$$w^{(s)}$

In a vector space we can always find a set of vectors such that all
elements of the vector space can be written in exactly one way as
a linear combination of the vectors of the set. Such a set is called
a basis of the vector space. We will consider only vector spaces where
the bases have a finite number of elements. We denote a basis by:
\item e = $\begin{bmatrix}e^{(1)}\\
e^{(2)}\\
\vdots\\
e^{(n)}
\end{bmatrix}$

The scalars used in this linear combination are called the coordinates
of x with respect to the basis e:
\item co(x) = $x$ = ($c_{1}$, $c_{2}$,...,$c_{n}$) \ensuremath{\Leftrightarrow}
x = $\sum_{i=1}^{n}$$c_{i}$ $\odot$$e^{(i)}$

A function f is called a linear function of a vector space V over
a field F, if it has the following properties:
\item \ensuremath{\forall} x, y \ensuremath{\in} V : f(x + y) = f(x) + f(y) 
\item \ensuremath{\forall} $a$ \ensuremath{\in} F, \ensuremath{\forall}
x \ensuremath{\in} V : f($a$x) = $a$f(x)
\end{itemize}

\end{document}
